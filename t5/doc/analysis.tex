\section{Theoretical Analysis}
\label{sec:analysis}

In this section we will do the theoretical analysis of our circuit.
The transfer function corresponds to the following equation:
\begin{equation}
	T(s) = (R_1C_1s/(1 + R_1C_1s))*(1 + R_3/R_4)*(1/(1 + R_2C_2s)))
\end{equation}

in which, $s = (2\pi*f)*j$
The frequency of the low pass filter is:
\begin{equation}
	f_L*2\pi = 1/R_2C_2
\end{equation}

The frequency of the high pass filter is:
\begin{equation}
	f_H*2\pi = 1/R_1C_1
\end{equation}

taking into account the equations above, the central frequency can be expressed as:
\begin{equation}
	f_c = \sqrt{f_L*f_H}
\end{equation}

The equations for the input impedance is the following:
\begin{equation}
	Z_in = R_1 + 1/jw_cC_1
\end{equation}

The output impedance can be expressed as:
\begin{equation}
	Z_out = 1/(jw_cC_2 + (1/R_2))
\end{equation}
	
In the following graphics, we plotted the output voltage gain and phase functions:
\begin{figure}[H] \centering
\includegraphics[width=0.6\linewidth]{magnitude_response_gain.eps}
\caption{Output voltage gain plot}
\label{fig:rc4}
\end{figure}

\begin{figure}[H] \centering
	\includegraphics[width=0.6\linewidth]{phase_response_gain.eps}
	\caption{Phase plot}
	\label{fig:rc4}
\end{figure}

In the next table, we have, the frequencies, gain and impedances mentioned above, as well as, the merit figure: 

\begin{table}[H]
	\centering
	\begin{tabular}{|l|r|}
		\hline    
		{\bf Name} & {\bf Value [mA]} \\ \hline
		Input Impedance of first stage & 484.433630 \\  \hline 
Gain of first stage & -262.790895 \\  \hline 
Gain of first stage in decibels & 48.392206 decibels \\  \hline 
Output Impedance of first stage & 886.284816 \\  \hline 
Input Impedance of second stage & 8598.855359 \\  \hline 
Gain of second stage & 0.991948 \\  \hline 
Gain of second stage in decibels & -0.070225 decibels \\  \hline 
Output Impedance of second stage & 0.302173 \\  \hline 
total Input Impedance & 484.433630 \\  \hline 
total Output Impedance & 3.981969 \\  \hline 
total Voltage Gain & -250.018098 \\  \hline 
total Voltage Gain in decibels & 47.959429 \\  \hline 
Cut-off frequency & 16681005.372001 \\  \hline 
Bandwidth & 2949401.128402 \\  \hline 
Cust & 2102.200000 \\  \hline 
Merit & 0.021039 \\  \hline 

	\end{tabular}
	\caption{octave results}
	\label{tab:op}
\end{table}
