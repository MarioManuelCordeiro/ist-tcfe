\section{Theoretical Analysis}
\label{sec:analysis}

In this section, the circuit shown in Figure~\ref{fig:rc} is analysed
theoretically.


\subsection{envelope detector}

In this section we will start by analyzing the envelope detector, where t varies from 0 to 20 ms, by using the main equation we see below:
\begin{equation}
	V_6n(t) = V_xe^{-(t)/CR_eq}
\end{equation}

We can now plot the graph of the solution for t raging from 0 to 20 ms:

\begin{figure}[h!] \centering
\includegraphics[width=0.6\linewidth]{}
\caption{Natural solution}
\label{fig:rc1}
\end{figure}
	
\subsection{voltage regulator}

The voltage regulator theoretical analysis of this circuit, like with the envelope detector is done with the help of a key equation below:
\begin{equation}
	V_6n(t) = V_xe^{-(t)/CR_eq}
\end{equation}

After this, we can plot the following graphic:
\begin{figure}[h!] \centering
	\includegraphics[width=0.6\linewidth]{}
	\caption{Natural solution}
	\label{fig:rc1}
\end{figure}


\subsection{Error}

Lastly, we will analyze the error plot, which like in the simulation analysis will give us the deviation in our results.

Considering:
\begin{equation}
	V_c(t) = V_6(t) - V_8(t)
\end{equation} 



The graphics of the error is:


\begin{figure}[h!] \centering
\includegraphics[width=0.6\linewidth]{}
\caption{Magnitude plot}
\label{fig:rc4}
\end{figure}



