\section{Theoretical Analysis}
\label{sec:analysis}

In this section, the circuit shown in Figure~\ref{fig:rcm} is analyzed
theoretically, by two different methods, which use the characteristics of the components present in the circuit and one of the two Kirchhoff Laws.
For the mesh method we used Kirchhoff's voltage law and for the nodal method we used Kirchhoff's current law.
Besides the independent voltage sources and current sources, the other components of the circuit are, the resistors, which obey Ohm's law, and dependent voltage sources and current sources.

\section{Mesh Method}

\begin{figure}[h] \centering
\includegraphics[width=0.6\linewidth]{mesh_diagram.pdf}
\caption{Voltage driven serial RC circuit.}
\label{fig:rc}
\end{figure}


To determine the values of $V_a$ an $V_b$ by applying Kirchhoff's voltage law, 4 equations are required. So the first equation can be writen as:

\begin{equation}
  I_d = I_4.
  \label{eq:kvl}
\end{equation}

Applying Kirxhhoff's Voltage Law (KVL) to equation~(\ref{eq:kvl}) we get the following:
\begin{equation}
  I3 = -\frac{R_4I_1}{R_4 + R_6 + R_7K_c}.
\end{equation}

And
\begin{equation}
  V_a(t) = (R_1 + R_3 + R_4)I_1 + R_4I_3 R_3I_2.
  \label{eq:kvl2}
\end{equation}

Knowing that:

\begin{equation}
  I_2 = I_b,
  \label{eq:vo_sol}
\end{equation}

And
\begin{equation}
  v_b = \frac{I_b}{V_b} = \frac{I_2}{V_b}
  \label{eq:vo_nat}
\end{equation}

We get:

\begin{equation}
  I_1(R_3K_b) + I_2(R_3K_b -1) = 0 
\end{equation}

\begin{table}[h]
  \centering
  \begin{tabular}{|l|r|}
    \hline    
    {\bf Name} & {\bf Value [A or V]} \\ \hline
    @I1[i] & 0.240224 \\ \hline 
@I2[i] & -0.251168 \\ \hline 
@I3[i] & 0.968913 \\ \hline 
@I4[i] & 1.038991 \\ \hline 
@G1[i] & -0.251168 \\ \hline 
@Id[current] & 1.038991 \\ \hline 
@R1[i] & 0.240224 \\ \hline 
@R2[i] & -0.251168 \\ \hline 
@R3[i] & -0.010944 \\ \hline 
@R4[i] & 1.209137 \\ \hline 
@R5[i] & -1.290158 \\ \hline 
@R6[i] & 0.968913 \\ \hline 
@R7[i] & 0.968913 \\ \hline
  \end{tabular}
  \caption{Operating point. A variable preceded by @ is of type {\em current}
    and expressed in Ampere; other variables are of type {\it voltage} and expressed in
    Volt.}
  \label{tab:op}
\end{table}

\section{Nodal Method}\label{sec:frequency-response}

In the nodal method, the circuit requires 6 equations to determine the values of V2, V3, V4, V5, V6, V7, which are obtained by applying Kirchhoff's current Law and a relation between the potential difference of two nodes and two other nodes, whose potential difference is imposed by a dependent voltage source. 

Since we considered $V_0$ to be the ground, which means that $V_o$ = 0 and therefore:
\begin{equation}
	V_i = V_a.
	\label{eq:kvl}
\end{equation}

We get the following equations by applying Kirchhoff's current law at different nodes:

For node 2 we have:
\begin{equation}
	0 = V_1\frac{-1}{R_2} + V_1(\frac{1}{R_1} + \frac{1}{R_2} + \frac{1}{R_3}) + V_5\frac{-1}{R_2} + V_4\frac{-1}{R_3}.
\end{equation}

Node 3:
\begin{equation}
	0 = V_1(\frac{-1}{R_2} - K_b) + V_3\frac{-1}{R_2} + V_4(K_b).
\end{equation}

Node 5:
\begin{equation}
	I_d = V_2(K_b) + V_4(\frac{-1}{R_5} - K_b) + V_5\frac{1}{R_5}.
\end{equation}

Node 6:
\begin{equation}
	0 = V_6(\frac{1}{R_6} - \frac{1}{R_7}) + V_7\frac{-1}{R_7} .
\end{equation}

Node 7:
\begin{equation}
	0 = V_4 + V_6(\frac{K_c}{R_6}) - V_7 .
\end{equation}

And finally for node 4:
\begin{equation}
	-I_d = V_2\frac{-1}{R_3} + V_4(\frac{1}{R_3} + \frac{1}{R_4} + \frac{1}{R_5}) + V_5\frac{-1}{R_5} + V_6\frac{-1}{R_7} + V_7\frac{1}{R_7}.
\end{equation}

\begin{table}[h]
  \centering
  \begin{tabular}{|l|r|}
    \hline    
    {\bf Name} & {\bf Value [A or V]} \\ \hline
    V(1) & 5.239365 \\  \hline 
V(2) & 4.988788 \\  \hline 
V(3) & 4.482071 \\  \hline 
V(4) & 5.023118 \\  \hline 
V(5) & 8.995714 \\  \hline 
V(6) & -1.962948 \\  \hline 
V(7) & -2.972813 \\  \hline 
V(8) & -1.962948 \\  \hline
  \end{tabular}
  \caption{Operating point. A variable preceded by @ is of type {\em current}
    and expressed in Ampere; other variables are of type {\it voltage} and expressed in
    Volt.}
  \label{tab:op}
\end{table}

